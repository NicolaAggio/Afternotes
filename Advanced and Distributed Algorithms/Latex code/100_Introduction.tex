\section{Introduction}

\subsection{Course content}
The course provides basic techniques for the development and analysis of advanced algorithms. The contents of the course are the following:
\begin{itemize}
    \item Revision of NP-completeness concepts;
    \item Approximation algorithms;
    \item Local search techniques;
    \item Randomized algorithms;
    \item Distributed algorithms;
    \item Examples of applications in security and robotics.
\end{itemize}

\subsection{Easy and hard problems}
In this section we provide a very informal definition of \textit{easy} and \textit{hard} problems. A problem is defined as \textbf{easy} if a computer is able to solve it in a very fast way; on the other hand, a problem is \textbf{hard} when even the fastest computers cannot solve it in a \textit{smart way}, i.e. they have to check all the possible solutions.

In general, when we approach a new problem, it is always very important to classify it as an \textit{easy} or \textit{hard} one: we will see two examples of similar problems, one of which is easy, and the other one is hard.

\subsubsection{Easy problem}
An example of easy problem is the problem of the \textbf{Könisberg} (Prussian city) \textbf{bridges}, introduced by Euler, which states the following: starting from an area, is it possible to come back to the same area by visiting all bridges exactly one time?

Before addressing more specifically this problem, we need to introduce some basic definitions. 

\definition{Path}{Given a graph and two nodes $x$ and $y$ of such graph, a \textbf{path} from $x$ to $y$ is a sequence of adjacent nodes that connects node $x$ to node $y$.}

\definition{Closed path}{A \textbf{closed path} is a path that starts from a node and goes back to the same node.}

\example{\imageB{intro1.png}{0.35}In the image, if we choose $x = 1$ and $y = 5$, then \{1,2,5\} is one of the paths from 1 to 5, while \{1,4,2,1\} is a closed path for $x=1$.}

\definition{Simple path}{A \textbf{simple path} is a path without repeated nodes.}

\definition{Circuit}{A \textbf{circuit} is a simple closed path.}

\example{In the previous image, \{1,2,5\} is a simple path, \{1,2,4,1,2,5\} is not a simple path and \{1,2,4,1\} is a circuit.}

\definition{Tree}{A \textbf{tree} is an undirected simple graph G that satisfies any of the following equivalent conditions: \begin{itemize}\item An acyclic graph with $|E|=|V|-1$; \item Connected and acyclic graph; \item Acyclic and connected graph, and by adding an edge it becomes cyclic.\end{itemize} A rooted tree is a tree with a node called root. A degree of a vertex of an undirected graph is the number of \textit{in-degree} and \textit{out-degree} edges. In each tree nodes with degree $\geq 2$ are called internal nodes, the ones with degree 1 are called leaves.}

Coming back to the problem of the Könisberg bridges, it is clear that such problems can be formulated using graphs, where we have a node for each area, and an arc for each bridge, as showed in the image.

\imageB{intro2.png}{0.3}

Intuitively, it easy easy to notice that in order to provide a solution to the problem, we require each node to have an even number of arcs (if we go away from a node, we need another arc to come back), but mathematically the previous problem can be re-formulated as the problem of searching for an Eulerian circuit in the graph.

\definition{Eulerian circuit}{A \textbf{Eulerian circuit} is a circuit that visits each \textbf{arc} of the graph \textbf{exactly once}.}

In order to solve this problem, we must rely on the following theorem.

\theoremBox{A graph is Eulerian (i.e. it has a Eulerian circuit) if and only if \begin{enumerate}\item Is is connected (i.e. there exists a path between every pair of nodes); \item All nodes have an even degree. \end{enumerate}}

Exploiting this theorem, we can easily solve the previous problem: \textit{If a graph does not satisfy the previous properties, then it is not Eulerian, otherwise we need an algorithm for finding it}. 

The algorithm that finds an Eulerian circuit in a graph is outlined as follows.

\begin{algorithm} \caption{Find the Eulerian circuit of a graph}
  \KwIn{Vertex $v_1$}
  \KwOut{..}
  
  \If{$v_1$ does not have outgoing arcs}{
    \Return $v_1$\;
  }
  \Else{
    Create a closed path $C = [v_1, v_2, .., v_k, v_1]$ from $v_1$ visiting all the arcs only once\;
    
    Erase all the arcs of path $C$\;
    
    \Return (\textit{Eulero}($v_1$), .., \textit{Eulero}($v_k$), $v_1$)\;
  }
\end{algorithm}

Note that the \textbf{complexity} of the algorithm is $O(m)$ steps, where $m=|E|$ is the number of arcs.

\subsubsection{Hard problem}
An example of \textit{hard problem} is the one of finding an Hamiltonian circuit in the graph.

\definition{Hamiltonian circuit}{An \textbf{Hamiltonian circuit} is a circuit that visits all the \textbf{nodes} of the graph \textbf{exactly once}.}

This is a \textit{hard} problem, since there exists no efficient algorithm that solves it in general (for any possible graph). The only solution is to find all paths.

As we can see, the two problems we've seen are very \textbf{similar} (in one we have to visit all the arcs exactly once, in the other all the nodes), but one is \textit{easy} to solve, the other is \textit{hard}.

Another hard problem similar to the search for the Hamiltonian circuit is the \textbf{travelling salesman problem}: given a weighted (with weight on arcs) complete (with all possible arcs) graph, find a Hamiltonian circuit of minimum weight. The salesman has to start from a city, visit all other cities, go back to the starting point travelling the minimal number of kilometers.

In general, computer science studies real problems, builds mathematical models and searches for efficient algorithms that solve the problems. It is useful to CLASSIFY the problems, i.e., to evaluate if a problem is “easy” (thus solvable by an efficient algorithm) or hard. In the second case we will search for: approximate solutions, or solutions which are correct with high probability, etc..

\subsection{Exercises}
\begin{enumerate}
    \item Execute the algorithm for finding Eulerian circuit in the following graph.

    \imageB{intro3.png}{0.25}
    
\end{enumerate}